\input{setup} % sets everything up.

\begin{document}
\include{titlepage}

\begin{frame}[label=toc]{Seminar Overview}
    \tableofcontents
\end{frame}

\section[Part V: ODEs]{Part V: Ordinary Differential Equations}
\begin{frame}{Introduction to ODEs}
    ODEs are functions defined by the relation between themselves and their derivatives. \pause
    As such, it is common for their derivatives to in some way resemble the original function. \pause
    The function types that allow this most easily are sine/cosine functions and exponentials. \pause
    
    \begin{exampleblock}{Examples}
        \begin{itemize}
            \item \(y' + y = 0.\) \pause
            \item \(\displaystyle \frac{dy}{dx} + y = 0\). \pause
            \item \(\displaystyle y'' + y' + y = 0\).
        \end{itemize}
    \end{exampleblock}
\end{frame}

\begin{frame}{Introduction to ODEs}
    \begin{block}{Types of ODEs}
    \begin{enumerate}
        \item \textbf{First order ODE}: highest derivative order is 1. Three forms:
        
        \begin{itemize}
            \uncover<1->{
            \item \only<1>{\textit{separable}}
            \only<2>{\textcolor{red}{\textit{separable}}}
            \only<3->{\textit{separable}}
            }
            
            \uncover<1->{
            \item \only<1-2>{\textit{linear}}
            
            \only<3>{\textcolor{red}{\textit{linear}}}
            
            \only<4->{\textit{linear}}
            }
            
            \uncover<1->{
            \item \only<1-3>{\textit{substitution}}
            
            \only<4>{\textcolor{red}{\textit{substitution}}}
            
            \only<5->{\textit{substitution}}
            }
        \end{itemize}
        
        \item \textbf{Second order ODE}: highest derivative order is 2. Two forms:
        \begin{itemize}
            \uncover<1->{
            \item \only<1-4>{\textit{homogeneous}}
            
            \only<5>{\textcolor{red}{\textit{homogeneous}}}
            
            \only<6->{\textit{homogeneous}}
            }
            \uncover<1->{
            \item \only<1-5>{\textit{non-homogeneous}}
            
            \only<6>{\textcolor{red}{\textit{non-homogeneous}}}
            }
        \end{itemize}
    \end{enumerate}
    \end{block}
    
    \uncover<2->{
    \begin{exampleblock}{Form of ODE}
        \only<2-2>{\[ y' = P(x)Q(y).\]}
        \only<3-3>{\[ y' + P(x)y = Q(x).\]}
        \only<4-4>{\[P(x)y' = Q(x) + G(y).\]}
        \only<5-5>{\[a y'' + b y' + c y = 0.\]}
        \only<6-6>{\begin{align*}ay'' + by' + cy' &= f(x), \\ y'' + P(x)y' + Q(x)y &= g(x).\end{align*}}
    \end{exampleblock}}
\end{frame}

\begin{frame}{First order ODE I: Separable ODE}
    \begin{block}{Method of solution} \pause
    \begin{enumerate}
        \item Separate variables so both sides contain one variable. \pause
        
        \item Integrate both sides with respect to said variable. \pause
        
        \item Rearrange for \(y\) as a function of \(x\) (or any equivalent form). \pause
    \end{enumerate}
    \end{block}
    
    \begin{exampleblock}{Example: (Paper)}
    \end{exampleblock}
\end{frame}

\begin{frame}{First order ODE II: Linear ODE}
    \begin{block}{Method of solution}
    \[\frac{dy}{dx} + P(x) y = Q(x).\]
    \pause
    \begin{enumerate}
        \item Identify expressions for \(P(x)\) and \(Q(x)\). \pause
        
        \item Create the \textbf{integrating factor} given by \[ R(x) = e^{\int P(x)\,dx}\] \pause
        
        \item Solution is of the form \[ y(x) = \frac{1}{R(x)} \int R(x) Q(x)\,dx.\]
    \end{enumerate}
    \end{block}
\end{frame}

\begin{frame}{First order ODE II: Linear ODE}
    \begin{exampleblock}{Example: (Paper)}
    \end{exampleblock}
\end{frame}

\begin{frame}{First order ODE III: Substitution ODE}
    \begin{block}{Method of solution}
    \[ P(x) \frac{dy}{dx} = Q(x) + G(y).\]
    \pause
    \begin{enumerate}
        \item Substitution is given in the question (use \(v = f(x, y)\)). \pause
        
        \item ODE becomes \[ \frac{dy}{dx} = \frac{Q(x)}{P(x)} + \frac{G(y)}{P(x)}.\] \pause
        
        \item Manipulate \(v\) to get expressions similar to \(Q(x)/P(x)\) and \(G(y)/P(x)\). \pause
        
        \item Result becomes either a separable or linear ODE.
    \end{enumerate}
    \end{block}
\end{frame}

\begin{frame}{First order ODE III: Substitution ODE}
    \begin{exampleblock}{Example: (Paper)}
    \end{exampleblock}
\end{frame}

\begin{frame}{Second order ODE I: Homogeneous}
    \begin{block}{Method of solution}
    \[ a\frac{d^2y}{dx^2} + b\frac{dy}{dx} + cy = 0.\]
    \pause
    \begin{enumerate}
        \item Obtain a \textbf{characteristic polynomial}: \(\displaystyle a\lambda^2 + b\lambda + c = 0.\) \pause
        
        \item Solve the quadratic for values of \(\lambda\). \pause
        
        \begin{itemize}
            \item If the values of \(\lambda\) are \textbf{distinct}, the solution is of the form \(y(x) = c_1 \cdot e^{\lambda_1 x} + c_2 \cdot e^{\lambda_2 x}\). \pause
            
            \item If the value of \(\lambda\) is \textbf{repeated}, the solution is of the form \(\displaystyle y(x) = c_1 \cdot e^{\lambda x} + c_2 \cdot xe^{\lambda x}\). \pause
            
            \item If the values of \(\lambda\) are \textbf{complex} (let \(\lambda = \alpha + i\beta\)), then the solution is of the form \(\displaystyle y(x) = e^{\alpha x}\left[c_1\cos(\beta x) + c_2\sin(\beta x)\right]\).
        \end{itemize}
    \end{enumerate}
    \end{block}
\end{frame}

\begin{frame}{Second order ODE I: Homogeneous}
    \begin{exampleblock}{Example: (paper)}
    \end{exampleblock}
\end{frame}

\begin{frame}{Second order ODE II: Non-homogeneous (undetermined coefficients)}
    \begin{block}{Method of solution}
        \[ a \frac{d^2y}{dx^2} + b \frac{dy}{dx} + cy = f(x).\]
        \pause
        \begin{enumerate}
            \item Solve the homogeneous differential equation to find the form of the \textbf{homogeneous solution} \(y_H(x)\). \pause
            
            \item Guess a suitable form for \(y_p(x)\). \pause
            \begin{itemize}
                \item If \(f(x)\) has an \textit{exponential term}, include it in the assumed form. For example: if \(f(x) = 3e^{3x}\), then guess \(Ae^{3x}\). \pause
                
                \item If \(f(x)\) has a \textit{polynomial term}, include \textbf{all lower degree terms} of the polynomial. \pause
                
                \item If \(f(x)\) has a \textit{sinusoidal}, include a sum of \textbf{both} \(\sin\) and \(\cos\) with the same frequency but different amplitudes.
            \end{itemize}
            
            \seti
        \end{enumerate}
    \end{block}
\end{frame}

\begin{frame}{Second order ODE II: Non-homogeneous (undetermined coefficients)}
    \begin{block}{Method of solution}
        \[ a \frac{d^2y}{dx^2} + b \frac{dy}{dx} + cy = f(x).\]
        \begin{enumerate}
        \conti
            
            \item Use your particular solution guess and substitute it into the differential equation and equate it to \(f(x)\). \pause
            
            \item Find common terms between each side and obtain a system of equations in terms of the undetermined coefficients. \pause
            
            \item Solution is the sum of its homogeneous and particular solution: \(\displaystyle y(x) = y_H(x) + y_p(x)\).
        \end{enumerate}
    \end{block}
\end{frame}

\begin{frame}{Second order ODE II: Non-homogeneous (undetermined coefficients)}
    \begin{exampleblock}{Example: (paper)}
    \end{exampleblock}
\end{frame}

\begin{frame}{Second order ODE III: Non-homogeneous (variation of parameters)}
    \begin{block}{Method of solution}
        \[ a \frac{d^2y}{dx^2} + b \frac{dy}{dx} + cy = f(x).\]
        \pause
        
        \begin{enumerate}
            \item Solve the homogeneous differential equation to find the form of the \textbf{homogeneous solution} \(y_H(x)\). \pause
            \begin{itemize}
                \item The homogeneous solution will be of the form \[ y_H(x) = c_1 \cdot y_1(x) + c_2 \cdot y_2(x).\] \pause
            \end{itemize}
            
            \item Evaluate the \textbf{Wronskian}, which is of the form \[ W(x) = \begin{vmatrix}y_1(x) & y_2(x) \\ y_1'(x) & y_2'(x)\end{vmatrix}.\]
            \seti
        \end{enumerate}
    \end{block}
\end{frame}

\begin{frame}{Second order ODE III: Non-homogeneous (variation of parameters)}
    \begin{block}{Method of solution}
        \[ a \frac{d^2y}{dx^2} + b \frac{dy}{dx} + cy = f(x).\]
        \pause
        
        \begin{enumerate}
            \conti
            \item The \textbf{particular solution} is then of the form \[ y_p(x) = -y_1(x) \int \frac{y_2(x) f(x)}{W(x)}\;dx + y_2(x) \int \frac{y_1(x) f(x)}{W(x)}\;dx.\] \pause
            
            \item The solution is the sum of its homogeneous and particular solution: \(\displaystyle y(x) = y_H(x) + y_p(x)\).
        \end{enumerate}
    \end{block}
\end{frame}

\begin{frame}{Second order ODE III: Non-homogeneous (variation of parameters)}
    \begin{exampleblock}{Example: (paper)}
        Use the method of variation of parameters to find the general solution of the differential equation \[ \frac{d^2y}{dx^2} - 2\frac{dy}{dx} + y = \frac{e^x}{x^3}.\]
    \end{exampleblock}
\end{frame}

\section{Part VI: Laplace transforms}
\section{Part VII: Fourier series}
\begin{frame}{Introduction to Fourier series}
    The \textbf{Fourier transform} describes the process of \textcolor{red}{approximating a \textbf{periodic function}} using a series of \textcolor{red}{trigonometric functions}. \pause Fourier transforms are defined by the following equations. \pause
    \begin{block}{} \pause
        \begin{align*}
            \uncover<4->{f(x) &= a_0 + \sum_{n = 1}^\infty a_n \cos\left(\frac{n\pi x}{L}\right) + \sum_{n = 1}^\infty b_n \sin \left(\frac{n \pi x}{L}\right),} \\
            \uncover<5->{a_0 &= \frac{1}{2L} \int_{-L}^L f(x)\;dx.} \\
            \uncover<6->{a_n &= \frac{1}{L} \int_{-L}^L f(x) \cos \left(\frac{n\pi x}{L}\right)\;dx.} \\
            \uncover<7->{b_n &= \frac{1}{L} \int_{-L}^L f(x) \sin \left(\frac{n\pi x}{L}\right)\;dx.}
        \end{align*}
    \end{block}
    \uncover<8->{\(n\) represents the \textbf{number of iterations} for the series, and \(L\) represents the \textbf{half period} of the function.}
\end{frame}

\begin{frame}{Even and Odd integrals}
    The \textbf{Fourier transformation process} evidently involves the evaluation of \textit{integrals} \(-\) the process can be made simpler by using the arithmetic properties of even and odd integrals.
    
    \begin{block}{Properties of even and odd integrals}
        \begin{align*}
            \int_{-L}^L f(x) \;dx =
            \begin{cases}
                \displaystyle 2 \int_0^L f(x)\;dx & \text{if } f(x) \text{ is even} \\
                0 & \text{if } f(x) \text{ is odd}
            \end{cases}
        \end{align*}
        
        \begin{itemize}
            \item If a function \(f\) is \textbf{even}, then \(f(x) = f(-x)\).
            \item If a function \(f\) is \textbf{odd}, then \(f(x) = -f(-x)\).
        \end{itemize}
    \end{block}
\end{frame}
\section[Part VIII: PDEs]{Part VIII: Partial Differential Equations}
\end{document}
